\chapter{Seznámení s aplikací}
\label{3-seznameni-s-aplikaci}

\section{Navázání na projekt FGIS}

Tato diplomová práce navazuje na započatý projekt z předmětu Free Software GIS. Na projektu jsem se účastnil já, Tomáš Lauwerys a Michal Zíma. Projekt měl za úkol vytvořit jednoduchou webovou aplikaci, kdy přihlášení uživatelé mohli zobrazovat, vytvářet, editovat a mazat dat z databáze. Databáze pro tento projekt byla pouze cvičná a obsahovala pouze několik tabulek s textovými poli. Projekt ovšem nebyl zcela dokončen, protože se nepovedlo vytvořit všechny funkce tak, aby správně fungovali. Data do databáze šla přidávat, dali se zobrazovat a také mazat, ale při editaci se po uložení nepřepsala aktuální data v databázi. 

\begin{figure}[H] \centering
    \includegraphics[width=400pt]{./pictures/4-nahled-menu-fgis.PNG}
    \caption[Náhled aplikace vytvořené v projektu FGIS]{Náhled aplikace vytvořené v projektu FGIS}
	\label{fig:Náhled aplikace}              
\end{figure}

 \newpage
 
 \section{O aplikaci}

V aplikaci jsou použity pohledy založeny na třídách, které se v Pythonu používají pro nahrazení pohledů jako funkcí. Použity jsou základní pohledy jako ListView pro zobrazení většího obsahu tabulky nebo CreateView pro uložení dat do databáze. V aplikaci v soubory \emph{views.py} jsou tedy vytvořeny třídy pohledů pro každou url adresu. V souboru \emph{urls.py} jsou propojeny url adresy, pohledy a html šablony, na které se odkazují. Složka \emph{templates} poté obsahuje html soubory jednotlivých zobrazovaných stránek. Dále je v aplikaci v \emph{models.py} zobrazena struktura celé databáze. 

V adresáři projektu je pak v \emph{settings.py} nastaveno připojení k databázi, importována django aplikace, připojen adresář obsahující složku \emph{templates} nebo nastavení domovské adresy. Soubor \emph{urls.py} kromě administrátorské stránky také obsahuje připojení aplikace, kde se tedy odkazuje na soubor \emph{urls.py} v adresáři aplikace. 

\textcolor{red}{přiložit obrázky souborů?}


