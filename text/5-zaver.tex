\chapter{Závěr}
\label{5-zaver}

% \section{Závěr}

Cílem práce bylo vytvořit webovou aplikaci pro projekt
Viskalia. Aplikace měla být vytvořena tak, aby uměla zobrazovat,
vytvářet, editovat a mazat data v databázi. Tento základní předpoklad
byl splněn, kdy aplikace splňuje všechny tyto funkcio\-nality i s
náležitými úpravami.

Co se týče aplikace, je zde několik možností na zlepšení, popřípadě
dodělání nedostatků aplikace. Jedním z nedostatků je pracná tvorba
vytváření uživatelů a nastavování jejich práv k objektům, kdy se
každému uživateli musí samostatně nastavit práva ke každému objektu,
pro který by měl mít právo editace. Toto by šlo vyřešit jednoduchým
skriptem pro hromadný zápis práv do tabulky managers. Dalším
nedostatkem je použití balíčku pro editaci administrátorského
%% ML: Preformulujte vetu "Aktualni..."
prostředí admin\_interface. Aktuální problém chybějící šablona, která
by se při sestavování kontejnerů vložila do databáze. Při opětovném
vytvoření se tedy vždy restartuje nastavení, které superuživatel
vytvoří. Možným řešením je tuto šablonu vytvořit a přidat jí při
spouštění do tabulky admin\_interface, nebo toto rozšíření odstranit a
upravit základní HTML šablonu, kterou poskytuje Django. Posledním
námětem je dokončení uživatelské aplikace, aby byla uživatelsky
přívětivější, ať už se jedná pouze o zobrazování dat, nebo dodělání
funkcionalit podobných administrátorskému rozhraní.

Jedním z úkolů projektového týmu Viskalia bylo také automatizovat
nasazení aplikace a databáze. Tento problém byl také vyřešen, ovšem
obsahuje několik nedostatků, které by bylo potřeba opravit. Prvním
nedostatkem je synchronizace struktury databáze a vložených dat, kdy
při tvorbě tabulky, která přebírá data z centrální databáze není
proveden sken této databáze, ale je přímo vytvořena tabulka, která
odpovídá její aktuální verzi. Pokud se tedy provedou změny v této
databázi, nepromítnou se do námi vytvořená tabulky. Dalším problémem
je stahování obrázků z databázového serveru. Obrázky nejsou stahovány
z databázového serveru, ale je stahován pouze předpřipravený soubor s
několika vybranými obrázky.
