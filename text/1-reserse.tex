\chapter{Rešerše}
\label{1-reserse}

\section{Volba metody pro tvorbu webové aplikace}


Při volbě, jakou metodu pro tvorbu aplikace použít, se nabízely dvě variant. První variantou by bylo použit skriptovací jazyka PHP a druhou použít python framework pro tvorbu webových aplikací. 

Hlavním faktorem při volbě metody zde byla úplná neznalost jazyka PHP, se kterým jsem se během studia ani mimo něj nesetkal. Jasnou volbou tedy bylo použití některého z python frameworků. 

\section{Volba python frameworku}
Výběr vhodného frameworku po tvorbu aplikace byl vzhledem k jejich množství opravdu složitý. Mezi hlavní kritéria pro výběr frameworku jsme zařadili snadnou práce s databází, implementovat per object permissions a dostupnost výukových materiálů a komunitní podpora. Nakonec se vybíralo se ze dvou frameworků, a to Flask a Django.

\begin{figure}[H] \centering
    \includegraphics[width=240pt]{./pictures/1-django-vs-flask.jpeg}
    \caption[Flask vs Django]{Flask vs Django \cite{}}
	\label{fig:Flask vs Django}                                
\end{figure}

Flask je open source webový framework napsaný v Pythonu, klasifikovaný jako mikro z důvodu, že není potřeba žádných dodatečných knihoven ani jiných nástrojů. Do Flasku lze ovšem doinstalovat různá rozšíření, která v základní verzi chybí jako například Flask-Admin, což je administrátorské rozhraní pro správu uživatelů a objektů v databázi. Výhodou je jeho velká obliba v komunitě, kdy v roce 2020 měl druhé místo na GitHubu z webových frameworků, a díky tomu disponuje spoustou materiálů a tutoriálů. Avšak velká nevýhoda flasku pro vývoj naší aplikace je, že nedisponuje data modely. Pokud bychom v průběhu vývoje chtěli přidat sloupec do naší databáze, musíme to udělat ručně v databázi a poté ho přidat do třídy ve webové aplikaci.

Django je nejpoužívanějším open source frameworkem disponuje objektově relačním mapováním. Jeho skvělou vlastností je tedy možnost generovat model z databáze, který se zde může upravovat a poté jednoduchými příkazy promítnout úpravy do databáze. Další výhodou je implementované administrátorské rozhraní a možnost doinstalováním přídavných balíčků, poskytující jak grafické úpravy, tak přidané funkce. Jedním z nich je i balíček django-guardian zajišťující podporu per object permissions.

Z těchto dvou frameworků jsem nakonec vybral pro tvorbu aplikace Django. To oproti Flasku disponovalo snadnou a rychlou práci s modelem databáze. Další výhodou je vývojáři implementované administrátorské rozhraní, které se nemusí doinstalovat pomocí přídavného balíčku. Z hlediska popularity a dostupnosti výukových materiálu jsou na tom oba frameworky podobně, kdy se oba řadí na první dvě místa výrazně před ostatní frameworky. 

\textbf{}
\textit{}



















