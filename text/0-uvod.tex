\chapter*{Úvod}
\addcontentsline{toc}{chapter}{Úvod}
\markboth{ÚVOD}{}
\label{0-uvod}

%% ML: stale jste neopravil sjednoceni casu, v textu se prolina
%% pritomny cas s budouci (dokonce v jedne kapitole!). Typicky se pise
%% v minulem case (z pohledu ctenare). Podstatne je ale zvolit stejny
%% cas v celem textu prace. Zvolte variantu tak, abyste nemusel
%% prepisovat cely text.

Cílem práce bylo vytvořit webovou aplikaci pro projekt VISKALIA
(Virtuální skansen lidové architektury), identifikační kód projektu:
DG20P02OVV003. Projekt se zaměřuje na záchranu fondů plánové, kresebné
a fotografické dokumentace lidové architektury v ČR. Jeho smyslem je
zpřístupnit tyto fondy širší veřejnosti. Toho se chce docílit
vytvořením virtuálního skansenu, který bude obsahovat zejména 3D
modely architektonických památek lidového stavitelství na území
ČR. Dále bude vytvořena veřejná databáze mapových výstupů, plánů,
fotografií a dalších dokumentů obohacená o množství metadat. Bude
využívána především při vzdělávání, ale také prostřednictvím výstav a
publikací.

Aplikace je zaměřená na interní správu objektů a nebude tedy přístupná
veřejnosti. Měla by umět zobrazovat data z databáze převzaté
z fondů Národního muzea. Ta bude dále rozšířena o uživateli poskytovaná
data jako fotografie, modely, dokumenty a jejich metadata. Tato data
už budou moci uživatelé přidávat, zobrazovat, mazat a editovat. Jeden
z hlavních požadavků je také na oprávnění k objektům, tedy aby
uživatelé mohli editovat pouze taková data, ke kterým mají udělené
oprávnění. Úkolem projektového týmu je také zajistit jednotný vývoj a nasazení 
aplikace společně s databází.

V první kapitole práce se diskutuje o možných technologických řešeních práce jako 
je volba vývojového prostředí aplikace nebo databázového systému. Na to navazují 
technologie, použité jak při vývoji aplikace, tak pro úpravu databázové struktury nebo 
automatického nasazení aplikace. Dále je popsáno navázání na rozpracovanou aplikaci a
použité databáze. Poslední kapitola se zabývá nejen samotným vývojem aplikace, ale také 
normalizací databáze a použitím softwaru Docker, pro vytvoření jednotného prostředí pro 
vývoj aplikace.



