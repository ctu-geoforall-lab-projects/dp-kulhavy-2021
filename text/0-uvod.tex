\chapter*{Úvod}
\addcontentsline{toc}{chapter}{Úvod}
\markboth{ÚVOD}{}
\label{0-uvod}

%% ML: stale jste neopravil sjednoceni casu, v textu se prolina
%% pritomny cas s budouci (dokonce v jedne kapitole!). Typicky se pise
%% v minulem case (z pohledu ctenare). Podstatne je ale zvolit stejny
%% cas v celem textu prace. Zvolte variantu tak, abyste nemusel
%% prepisovat cely text.

Cílem práce je vytvořit webovou aplikaci pro projekt VISKALIA
(Virtuální skansen lidové architektury), identifikační kód projektu:
DG20P02OVV003. Projekt se zaměřuje na záchranu fondů plánové, kresebné
a fotografické dokumentace lidové architektury v ČR. Jeho smyslem je
zpřístupnit tyto fondy širší veřejnosti. Toho se chce docílit
vytvořením virtuálního skansenu, který bude obsahovat zejména 3D
modely architektonických památek lidového stavitelství na území
ČR. Dále bude vytvořena veřejná databáze mapových výstupů, plánů,
fotografií a dalších dokumentů obohacená o množství metadat. Bude
využívána především při vzdělávání, ale také prostřednictvím výstav a
publikací.

Aplikace bude zaměřená na interní správu objektů a nebude tedy přístupná
veřejnosti. Měla by umět zobrazovat data z databáze převzaté
%% ML: nepouzivejte "nami", specifikujte. Uz jsme to probirali...
z fondů Národního muzea. Ta bude dále rozšířena o námi poskytovaná
%% ML: Tato
data jako fotografie, modely, dokumenty a jejich metadata. Tyto data
už budou moci uživatelé přidávat, zobrazovat, mazat a editovat. Jeden
%% ML: ani to nemate opraveno... (per object permissions), pouzijte
%% cesky ekvivalent
z hlavních požadavků je také na per object permissions, tedy aby
uživatelé mohli editovat pouze taková data, ke kterým mají udělené
oprávnění. Aplikace bude po dokončení spuštěna na serveru ČVUT.

%% ML: Uvod je stale minimalisticky. Muzete pridat alespon odstavec
%% popisujici strukturu prace.

