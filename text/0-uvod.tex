\chapter*{Úvod}
\addcontentsline{toc}{chapter}{Úvod}
\markboth{ÚVOD}{}
\label{0-uvod}

%% ML: text prace se vetsinou pise v minulem cas (z pohledu ctenare)
%% ML: pokud v cele praci pouzivate pritomny cas, tak uz to tak
%% nechte, kazdopadne je treba to sladit
%% ML: neni potreba popisovat veskere cile projektu, pouze relevantni
%% cile pro vasi praci
Cílem práce je vytvořit webovou aplikaci pro projekt VISKALIA
%% uvest referenci na grant (cislo/referenci grantu)
(Virtuální skansen lidové architektury). Projekt se zaměřuje na
záchranu fondů plánové, kresebné a fotografické dokumentace lidové
architektury v ČR. Jeho smyslem je zpřístupnit tyto fondy širší
veřejnosti. Toho se chce docílit vytvořením virtuálního skansenu,
který bude obsahovat zejména 3D modely architektonických památek
lidového stavitelství na území ČR. Dále bude vytvořena veřejná
databáze mapových výstupů, plánů, fotografií a dalších dokumentů
obohacená o množství metadat. Bude využívána především při vzdělávání,
ale také prostřednictvím výstav a publikací.

Aplikace je zaměřená na interní správu objektů a nebude tedy přístupná
veřejností. Aplikace by měla umět zobrazovat data z databáze převzaté
z fondů Národního muzea. Ta bude dále rozšířena o námi poskytovaná
data jako fotografie, modely, dokumenty a jejich metadata. Tyto data
už budou moci uživatelé přidávat, zobrazovat, mazat a editovat. Jeden
z hlavních požadavků je také na per object permissions, tedy aby
uživatelé mohli editovat pouze taková data, ke kterým mají udělené
oprávnění. Aplikace bude po dokončení spuštěna na serveru ČVUT.


